\documentclass[a4paper,10pt]{article}
\usepackage[T1]{fontenc}
\usepackage{lmodern}
\usepackage{amssymb,amsmath}
\usepackage{ifxetex,ifluatex}
%\usepackage{fixltx2e} % provides \textsubscript
% use upquote if available, for straight quotes in verbatim environments
\IfFileExists{upquote.sty}{\usepackage{upquote}}{}
\ifnum 0\ifxetex 1\fi\ifluatex 1\fi=0 % if pdftex
  \usepackage[utf8]{inputenc}
\else % if luatex or xelatex
  \ifxetex
    \usepackage{mathspec}
    \usepackage{xltxtra,xunicode}
  \else
    \usepackage{fontspec}
  \fi
  \defaultfontfeatures{Mapping=tex-text,Scale=MatchLowercase}
  \newcommand{\euro}{€}
\fi
% use microtype if available
\IfFileExists{microtype.sty}{\usepackage{microtype}}{}
\usepackage{color}
\usepackage{fancyvrb}
\newcommand{\VerbBar}{|}
\newcommand{\VERB}{\Verb[commandchars=\\\{\}]}
\DefineVerbatimEnvironment{Highlighting}{Verbatim}{commandchars=\\\{\}}
% Add ',fontsize=\small' for more characters per line
\newenvironment{Shaded}{}{}
\newcommand{\KeywordTok}[1]{\textcolor[rgb]{0.00,0.44,0.13}{\textbf{{#1}}}}
\newcommand{\DataTypeTok}[1]{\textcolor[rgb]{0.56,0.13,0.00}{{#1}}}
\newcommand{\DecValTok}[1]{\textcolor[rgb]{0.25,0.63,0.44}{{#1}}}
\newcommand{\BaseNTok}[1]{\textcolor[rgb]{0.25,0.63,0.44}{{#1}}}
\newcommand{\FloatTok}[1]{\textcolor[rgb]{0.25,0.63,0.44}{{#1}}}
\newcommand{\CharTok}[1]{\textcolor[rgb]{0.25,0.44,0.63}{{#1}}}
\newcommand{\StringTok}[1]{\textcolor[rgb]{0.25,0.44,0.63}{{#1}}}
\newcommand{\CommentTok}[1]{\textcolor[rgb]{0.38,0.63,0.69}{\textit{{#1}}}}
\newcommand{\OtherTok}[1]{\textcolor[rgb]{0.00,0.44,0.13}{{#1}}}
\newcommand{\AlertTok}[1]{\textcolor[rgb]{1.00,0.00,0.00}{\textbf{{#1}}}}
\newcommand{\FunctionTok}[1]{\textcolor[rgb]{0.02,0.16,0.49}{{#1}}}
\newcommand{\RegionMarkerTok}[1]{{#1}}
\newcommand{\ErrorTok}[1]{\textcolor[rgb]{1.00,0.00,0.00}{\textbf{{#1}}}}
\newcommand{\NormalTok}[1]{{#1}}
\ifxetex
  \usepackage[setpagesize=false, % page size defined by xetex
              unicode=false, % unicode breaks when used with xetex
              xetex]{hyperref}
\else
  \usepackage[unicode=true]{hyperref}
\fi
\hypersetup{breaklinks=true,
            bookmarks=true,
            pdfauthor={},
            pdftitle={},
            colorlinks=true,
            citecolor=blue,
            urlcolor=blue,
            linkcolor=magenta,
            pdfborder={0 0 0}}
\urlstyle{same}  % don't use monospace font for urls
\setlength{\parindent}{0pt}
\setlength{\parskip}{6pt plus 2pt minus 1pt}
\setlength{\emergencystretch}{3em}  % prevent overfull lines
\setcounter{secnumdepth}{0}

\title{GIT: Fast and Furious}
\author{Shahab Rezaee <\href{mailto:shahabrezaee@gmail.com}{shahabrezaee@gmail.com}>}
\date{}

\begin{document}

\maketitle

\section{What is Git?}\label{what-is-git}

Git is:

\begin{itemize}
\itemsep1pt\parskip0pt\parsep0pt
\item
  A Distributed version control, means:

  \begin{itemize}
  \itemsep1pt\parskip0pt\parsep0pt
  \item
    Each user maintain their own repository
  \end{itemize}
\item
  A tool that tracks changes instead of versions, specially for text
  content
\end{itemize}

Every single updates are \textbf{change sets} which shapes repository
state. There is no master repository but many working copies which
contains their own \textbf{change sets}. Example: All of the followings
may be valid:

\begin{verbatim}
  Repo 1: A -- B -- C -- D -- E -- F
  Repo 2: A -- B -- C
  Repo 3: A -- B -- E -- F
\end{verbatim}

So, there is no centralized copy on server tracking every movements!
\emph{\textbf{Goodbye dictators! Hello FREEDOM!}}

Triumphs:

\begin{itemize}
\itemsep1pt\parskip0pt\parsep0pt
\item
  No network access required
\item
  No single point of \emph{FAILIER}
\item
  Developers could works independently
\item
  Easier code review process
\end{itemize}

Git generally is not good for tracking binary files. So, storing many
binary objects in git is bad habit! (see Git LFS)

\section{Creation!}\label{creation}

\begin{Shaded}
\begin{Highlighting}[]
  \NormalTok{$ }\KeywordTok{cd} \NormalTok{path/to/repository/root/}
  \NormalTok{$ }\KeywordTok{git} \NormalTok{init}
\end{Highlighting}
\end{Shaded}

See \texttt{.git} directory!

Create your first files. Commit your changes.

\begin{Shaded}
\begin{Highlighting}[]
  \NormalTok{$ }\KeywordTok{touch} \NormalTok{first.txt}
  \NormalTok{$ }\KeywordTok{git} \NormalTok{add .}
  \NormalTok{$ }\KeywordTok{git} \NormalTok{commit -m }\StringTok{"Initial commit!"}
\end{Highlighting}
\end{Shaded}

\texttt{-m} means message. You could use \texttt{git commit} to access
editor to write multi line commit message. You could specify editor in
eather \texttt{\textasciitilde{}/.bashrc} or
\texttt{\textasciitilde{}/.bash\_profile} :

\begin{Shaded}
\begin{Highlighting}[]
\KeywordTok{export} \OtherTok{EDITOR=}\NormalTok{nano}
\end{Highlighting}
\end{Shaded}

Write good messages for commits.
(\href{http://chris.beams.io/posts/git-commit/}{How to write a Git
Commit Message}) Use meaningful conventions such as issue numbers,
support tickets, git refs and etc in message title.

Now, see current changes:

\begin{Shaded}
\begin{Highlighting}[]
  \NormalTok{$ }\KeywordTok{git} \NormalTok{log}
  \NormalTok{$ }\KeywordTok{git} \NormalTok{log --since=}\StringTok{"2016-06-04"}
  \NormalTok{$ }\KeywordTok{git} \NormalTok{log --until=}\StringTok{"2016-06-04"}
  \NormalTok{$ }\KeywordTok{git} \NormalTok{log --author=}\StringTok{"Shahab"}
  \NormalTok{$ }\KeywordTok{git} \NormalTok{log --grep=}\StringTok{"Init"}
\end{Highlighting}
\end{Shaded}

\begin{Shaded}
\begin{Highlighting}[]
  \NormalTok{$ }\KeywordTok{git} \NormalTok{help}
  \NormalTok{$ }\KeywordTok{git} \NormalTok{help init}
\end{Highlighting}
\end{Shaded}

\section{Git structure}\label{git-structure}

\subsection{Git levels}\label{git-levels}

\begin{verbatim}
    +--------------+
    |  repository  |
  / +--------------+
  |
  | git commit
  |
  \ +-----------------+
    |  staging index  |
  / +-----------------+
  |
  | git add
  |
  \ +---------------------+
    |  working directory  |
    +---------------------+
\end{verbatim}

\subsection{Change Sets}\label{change-sets}

\begin{itemize}
\itemsep1pt\parskip0pt\parsep0pt
\item
  \textbf{change set} is a single patch file may affects many files in
  repository.
\item
  Git generates a SHA-1 checksum for every \textbf{change set}.
\item
  SHA-1 guaranties that same data have a same checksum.
\item
  SHA-1 generate 40 character hexadecimal ({[}0-9a-f{]}) string.
\end{itemize}

\begin{verbatim}
  +------------+
  |  checksum  |
  +------------+
  |  parent    |
  +------------+
  |  author    |
  +------------+
  |  message   |
  +------------+
\end{verbatim}

Example: Consider followings are initial commits of repository:

\begin{verbatim}
A -- B -- C
\end{verbatim}

thus:

\begin{verbatim}
  A.parent == NULL
  B.parent == A.checksum
  C.parent == B.checksum
\end{verbatim}

\subsection{HEAD}\label{head}

HEAD is:

\begin{itemize}
\itemsep1pt\parskip0pt\parsep0pt
\item
  pointer to tip of current branch of repository
\item
  last state of repository checked out
\item
  parent of next commit
\end{itemize}

\begin{Shaded}
\begin{Highlighting}[]
  \NormalTok{$ }\KeywordTok{cd} \NormalTok{.git}
  \NormalTok{$ }\KeywordTok{ls} \CommentTok{# there is a file named HEAD}
  \NormalTok{$ }\KeywordTok{cat} \NormalTok{HEAD  }\CommentTok{# location of current head}
  \NormalTok{$ }\KeywordTok{cd} \NormalTok{refs/heads}
  \NormalTok{$ }\KeywordTok{cat} \NormalTok{master  }\CommentTok{# checksum of lastest commit}
\end{Highlighting}
\end{Shaded}

\section{Create changes!}\label{create-changes}

\begin{Shaded}
\begin{Highlighting}[]
  \NormalTok{$ }\KeywordTok{git} \NormalTok{status  }\CommentTok{# there is no changes}
  \NormalTok{$ }\KeywordTok{touch} \NormalTok{second.txt}
  \NormalTok{$ }\KeywordTok{touch} \NormalTok{third.txt}
  \NormalTok{$ }\KeywordTok{git} \NormalTok{status  }\CommentTok{# there are untracked files (changes not staged)}
  \NormalTok{$ }\KeywordTok{git} \NormalTok{add second.txt}
  \NormalTok{$ }\KeywordTok{git} \NormalTok{status  }\CommentTok{# second.txt staged to be commited, third.txt still untracked}
  \NormalTok{$ }\KeywordTok{git} \NormalTok{commit -m }\StringTok{"Add second.txt to the project"}
  \NormalTok{$ }\KeywordTok{git} \NormalTok{status  }\CommentTok{# third.txt is untracked}
  \NormalTok{$ }\KeywordTok{git} \NormalTok{log  }\CommentTok{# new commit is listed}
  \NormalTok{$ }\KeywordTok{git} \NormalTok{add third.txt}
  \NormalTok{$ }\KeywordTok{git} \NormalTok{commit -m }\StringTok{"Add third file to the project"}
  \NormalTok{$ }\KeywordTok{git} \NormalTok{log  }\CommentTok{# another commit is listed}
  \NormalTok{$ }\KeywordTok{echo} \StringTok{"a new line at the end"} \KeywordTok{>>} \NormalTok{first.txt}
  \NormalTok{$ }\KeywordTok{git} \NormalTok{status  }\CommentTok{# first.txt is modified. changes are tracked but not staged yet}
  \NormalTok{$ }\KeywordTok{git} \NormalTok{diff  }\CommentTok{# see difference between current working directory and HEAD}
  \NormalTok{$ }\KeywordTok{echo} \StringTok{"hellllllllllllllllllllo again!"} \KeywordTok{>>} \NormalTok{third.txt}
  \NormalTok{$ }\KeywordTok{git} \NormalTok{diff  }\CommentTok{# multiple file changes}
  \NormalTok{$ }\KeywordTok{git} \NormalTok{add first.txt}
  \NormalTok{$ }\KeywordTok{git} \NormalTok{diff  }\CommentTok{# Oh staged changes are not available}
  \NormalTok{$ }\KeywordTok{git} \NormalTok{diff --staged  }\CommentTok{# dada! see differences between staged index and HEAD (git 1.7+, --cached for prev.)}
  \NormalTok{$ }\KeywordTok{git} \NormalTok{add third.txt}
  \NormalTok{$ }\KeywordTok{git} \NormalTok{commit -m }\StringTok{"Minor bottom changes ;)"}
  \NormalTok{$ }\KeywordTok{touch} \NormalTok{to_be_deleted.txt}
  \NormalTok{$ }\KeywordTok{git} \NormalTok{commit -am }\StringTok{"Add files to delete later"}
  \NormalTok{$ }\KeywordTok{rm} \NormalTok{to_be_deleted.txt}
  \NormalTok{$ }\KeywordTok{git} \NormalTok{status  }\CommentTok{# changes not staged}
  \NormalTok{$ }\KeywordTok{git} \NormalTok{rm to_be_deleted.txt}
  \NormalTok{$ }\KeywordTok{git} \NormalTok{status  }\CommentTok{# delete is staged to be commited}
  \NormalTok{$ }\KeywordTok{git} \NormalTok{commit -m }\StringTok{"to_be_deleted is now deleted!"}
  \NormalTok{$ }\KeywordTok{touch} \NormalTok{return_of_to_be_deleted.txt}
  \NormalTok{$ }\KeywordTok{git} \NormalTok{commit -am }\StringTok{"To Be Deleted: The Return"}
  \NormalTok{$ }\KeywordTok{git} \NormalTok{rm return_of_to_be_deleted.txt}
  \NormalTok{$ }\KeywordTok{ls}  \CommentTok{# Horay! It's gone!}
  \NormalTok{$ }\KeywordTok{git} \NormalTok{status  }\CommentTok{# It's also staged!}
  \NormalTok{$ }\KeywordTok{git} \NormalTok{commit -m }\StringTok{"Fixing consistant to_be_deleted.txt bug"}
  \NormalTok{$ }\KeywordTok{mv} \NormalTok{first.txt primary.txt}
  \NormalTok{$ }\KeywordTok{git} \NormalTok{status  }\CommentTok{# seems first.txt is deleted (tracked), primary.txt is created (untracked)}
  \NormalTok{$ }\KeywordTok{git} \NormalTok{add primary.txt}
  \NormalTok{$ }\KeywordTok{git} \NormalTok{rm first.txt}
  \NormalTok{$ }\KeywordTok{git} \NormalTok{status  }\CommentTok{# OH! Shit! git got it! it's renamed!}
  \NormalTok{$ }\KeywordTok{git} \NormalTok{mv second.txt secondary.txt}
  \NormalTok{$ }\KeywordTok{git} \NormalTok{status  }\CommentTok{# Horay! renamed and staged!}
\end{Highlighting}
\end{Shaded}

\begin{Shaded}
\begin{Highlighting}[]
  \NormalTok{$ }\KeywordTok{git} \NormalTok{reset HEAD --hard  }\CommentTok{# we describe it later!}
  \NormalTok{$ }\KeywordTok{nano} \NormalTok{first.txt  }\CommentTok{# change something in one line}
  \NormalTok{$ }\KeywordTok{git} \NormalTok{diff first.txt}
  \NormalTok{$ }\KeywordTok{git} \NormalTok{diff --color-words first.txt}
\end{Highlighting}
\end{Shaded}

\section{Time machine! Get things back or OMG! I am in love with
Git!}\label{time-machine-get-things-back-or-omg-i-am-in-love-with-git}

\subsection{From working directory to
HEAD}\label{from-working-directory-to-head}

\subsubsection{Undo changes of one file}\label{undo-changes-of-one-file}

\begin{Shaded}
\begin{Highlighting}[]
  \NormalTok{$ }\KeywordTok{echo} \StringTok{"I gonna falllllll%@#%}\OtherTok{$#}\StringTok{^#%&"} \KeywordTok{>>} \NormalTok{first.txt}
  \NormalTok{$ }\KeywordTok{git} \NormalTok{status  }\CommentTok{# Next day when I check the repo: Ohhhh!}
  \NormalTok{$ }\KeywordTok{git} \NormalTok{checkout -- first.txt  }\CommentTok{# git checkout first.txt will work on many cases but what if we have a branch called 'first.txt'! god bless '--'}
\end{Highlighting}
\end{Shaded}

\subsection{From staged to working
directory}\label{from-staged-to-working-directory}

\begin{Shaded}
\begin{Highlighting}[]
  \NormalTok{$ }\KeywordTok{echo} \StringTok{"I gonna falllllll%@#%}\OtherTok{$#}\StringTok{^#%&"} \KeywordTok{>>} \NormalTok{first.txt }\KeywordTok{&&} \KeywordTok{git} \NormalTok{add first.txt}
  \NormalTok{$ }\KeywordTok{git} \NormalTok{status  }\CommentTok{# my god! they are staged!}
  \NormalTok{$ }\KeywordTok{git} \NormalTok{reset HEAD first.txt}
\end{Highlighting}
\end{Shaded}

\subsection{From committed to working
directory}\label{from-committed-to-working-directory}

\begin{Shaded}
\begin{Highlighting}[]
  \NormalTok{$ }\KeywordTok{echo} \StringTok{"I gonna falllllll%@#%}\OtherTok{$#}\StringTok{^#%&"} \KeywordTok{>>} \NormalTok{first.txt }\KeywordTok{&&} \KeywordTok{git} \NormalTok{commit -am }\StringTok{"sombjhgsv f hsafhi fbuf"}
  \NormalTok{$ }\KeywordTok{git} \NormalTok{status}
  \NormalTok{$ }\KeywordTok{git} \NormalTok{log  }\CommentTok{# HOLY CRAP! They are commited}
\end{Highlighting}
\end{Shaded}

Consider this:

\begin{verbatim}
  A -- B -- C -- D -- E
\end{verbatim}

We already know that each commit is depends to the previous commit. So
if we remove one, repository would loose its data integrety. However, we
may able to change the last commit!

\subsubsection{Change last commit}\label{change-last-commit}

\begin{Shaded}
\begin{Highlighting}[]
  \NormalTok{$ }\KeywordTok{git} \NormalTok{commit -amend -m }\StringTok{"A nice good morning message!"}
  \NormalTok{$ }\KeywordTok{git} \NormalTok{status  }\CommentTok{# Nothing!}
  \NormalTok{$ }\KeywordTok{git} \NormalTok{log  }\CommentTok{# message of last commit is looking nice! Note that checksum and date are change too.}
  \NormalTok{$ }\KeywordTok{echo} \StringTok{"one more other line"} \KeywordTok{>>} \NormalTok{first.txt}
  \NormalTok{$ }\KeywordTok{echo} \StringTok{"one more other line"} \KeywordTok{>>} \NormalTok{second.txt}
  \NormalTok{$ }\KeywordTok{git} \NormalTok{add first.txt}
  \NormalTok{$ }\KeywordTok{git} \NormalTok{commit -m }\StringTok{"Append one more line to first and second"}  \CommentTok{# oops! we forgot second one!}
  \NormalTok{$ }\KeywordTok{git} \NormalTok{add second.txt}
  \NormalTok{$ }\KeywordTok{git} \NormalTok{commit --amend --no-edit  }\CommentTok{# Save or change message! Done!}
\end{Highlighting}
\end{Shaded}

\subsubsection{Bring commited changes out of
commit}\label{bring-commited-changes-out-of-commit}

\begin{Shaded}
\begin{Highlighting}[]
  \NormalTok{$ }\KeywordTok{git} \NormalTok{checkout CHECKSUM -- first.txt}
  \NormalTok{$ }\KeywordTok{git} \NormalTok{status}
  \NormalTok{$ }\KeywordTok{git} \NormalTok{reset HEAD first.txt}
\end{Highlighting}
\end{Shaded}

\subsubsection{Revert last commit!}\label{revert-last-commit}

\begin{Shaded}
\begin{Highlighting}[]
  \NormalTok{$ }\KeywordTok{git} \NormalTok{checkout -- first.txt  }\CommentTok{# Back to last commit}
  \NormalTok{$ }\KeywordTok{git} \NormalTok{log}
  \NormalTok{$ }\KeywordTok{git} \NormalTok{revert CHECKSUM  }\CommentTok{# revert last commit: git revert HEAD}
  \NormalTok{$ }\KeywordTok{git} \NormalTok{log  }\CommentTok{# a revert commit!}
  \NormalTok{$ }\KeywordTok{git} \NormalTok{revert CHECKSUM  }\CommentTok{# try to revert to older commits may cause confilicts}
  \NormalTok{$ }\KeywordTok{git} \NormalTok{revert --abort  }\CommentTok{# return to last state}
\end{Highlighting}
\end{Shaded}

\subsection{To infinity\ldots{} and
beyond!}\label{to-infinity-and-beyond}

There are three kind of reset:

\begin{itemize}
\itemsep1pt\parskip0pt\parsep0pt
\item
  soft: no changes in niether staged index nor working directory
\item
  mixed (default): change staged index to match HEAD but apply no
  changes to the working directory
\item
  hard: change both staged index and working directory to match HEAD
\end{itemize}

\subsubsection{Soft reset}\label{soft-reset}

\begin{Shaded}
\begin{Highlighting}[]
  \NormalTok{$ }\KeywordTok{cat} \NormalTok{.git/HEAD}
  \NormalTok{$ }\KeywordTok{cat} \NormalTok{.git/refs/heads/master}
  \NormalTok{$ }\KeywordTok{git} \NormalTok{log}
  \NormalTok{$ }\KeywordTok{git} \NormalTok{reset --soft CHECKSUM}
  \NormalTok{$ }\KeywordTok{git} \NormalTok{log}
  \NormalTok{$ }\KeywordTok{git} \NormalTok{status}
  \NormalTok{$ }\KeywordTok{git} \NormalTok{diff --staged}
  \NormalTok{$ }\KeywordTok{cat} \NormalTok{.git/refs/heads/master}
  \NormalTok{$ }\KeywordTok{git} \NormalTok{reset --soft CHECKSUM_OF_LATESET  }\CommentTok{# it's not destructive}
\end{Highlighting}
\end{Shaded}

\subsubsection{Mixed reset}\label{mixed-reset}

\begin{Shaded}
\begin{Highlighting}[]
  \NormalTok{$ }\KeywordTok{git} \NormalTok{log}
  \NormalTok{$ }\KeywordTok{git} \NormalTok{reset --mixed CHECKSUM}
  \NormalTok{$ }\KeywordTok{git} \NormalTok{log}
  \NormalTok{$ }\KeywordTok{git} \NormalTok{status  }\CommentTok{# committed changes are unstaged now}
  \NormalTok{$ }\KeywordTok{git} \NormalTok{add .}
  \NormalTok{$ }\KeywordTok{git} \NormalTok{reset HEAD FILENAME}
  \NormalTok{$ }\KeywordTok{git} \NormalTok{reset --mixed CHECKSUM_OF_LATESET  }\CommentTok{# it's not destructive too}
\end{Highlighting}
\end{Shaded}

\subsubsection{Hard reset}\label{hard-reset}

\begin{Shaded}
\begin{Highlighting}[]
  \NormalTok{$ }\KeywordTok{git} \NormalTok{log}
  \NormalTok{$ }\KeywordTok{git} \NormalTok{reset --hard CHECKSUM}
  \NormalTok{$ }\KeywordTok{git} \NormalTok{log}
  \NormalTok{$ }\KeywordTok{git} \NormalTok{status  }\CommentTok{# Nothing here!}
  \NormalTok{$ }\KeywordTok{echo} \StringTok{"this would be gone!"} \KeywordTok{>>} \NormalTok{first.txt}
  \NormalTok{$ }\KeywordTok{git} \NormalTok{reset --hard HEAD}
  \NormalTok{$ }\KeywordTok{git} \NormalTok{status  }\CommentTok{# nothing here! may be destructive or painful!}
  \NormalTok{$ }\KeywordTok{git} \NormalTok{reset --hard CHECKSUM_OF_LATESET  }\CommentTok{# it's here again but we are lucky because it's not collected yet!}
\end{Highlighting}
\end{Shaded}

\subsubsection{Reset the Reset}\label{reset-the-reset}

\begin{Shaded}
\begin{Highlighting}[]
  \NormalTok{$ }\KeywordTok{git} \NormalTok{reflog  }\CommentTok{# git help reflog}
  \NormalTok{$ }\KeywordTok{git} \NormalTok{reset --hard HEAD~3}
  \NormalTok{$ }\KeywordTok{git} \NormalTok{log}
  \NormalTok{$ }\KeywordTok{git} \NormalTok{reflog}
  \NormalTok{$ }\KeywordTok{git} \NormalTok{reset --hard HEAD@}\DataTypeTok{\{1\}}
\end{Highlighting}
\end{Shaded}

\subsection{From dirty to clean!}\label{from-dirty-to-clean}

\begin{Shaded}
\begin{Highlighting}[]
  \NormalTok{$ }\KeywordTok{touch} \NormalTok{jerk1.txt}
  \NormalTok{$ }\KeywordTok{touch} \NormalTok{jerk2.txt}
  \NormalTok{$ }\KeywordTok{git} \NormalTok{status}
  \NormalTok{$ }\KeywordTok{git} \NormalTok{clean}
  \NormalTok{$ }\KeywordTok{git} \NormalTok{clean -n  }\CommentTok{# Test run of clean command}
  \NormalTok{$ }\KeywordTok{git} \NormalTok{add jerk1.txt}
  \NormalTok{$ }\KeywordTok{git} \NormalTok{clean -f  }\CommentTok{# jerk1.txt still here}
  \NormalTok{$ }\KeywordTok{git} \NormalTok{reset HAED jerk1.txt}
  \NormalTok{$ }\KeywordTok{git} \NormalTok{clean -f}
  \NormalTok{$ }\KeywordTok{git} \NormalTok{status}
\end{Highlighting}
\end{Shaded}

\section{Ignorance!}\label{ignorance}

\begin{Shaded}
\begin{Highlighting}[]
  \NormalTok{$ }\KeywordTok{touch} \NormalTok{to_be_ignored.txt}
  \NormalTok{$ }\KeywordTok{git} \NormalTok{status}
  \NormalTok{$ }\KeywordTok{touch} \NormalTok{.gitignore}
\end{Highlighting}
\end{Shaded}

.gitignore accept patterns in each line:

\begin{itemize}
\item
  Very basic regular experssins * ? {[}aeiou{]} {[}0-9{]}
\item
  \texttt{!} means NOT! Example: \texttt{*.jpg     !logo.jpg}
\item
  Triling slash ignores directoris: \texttt{static/}
\end{itemize}

\begin{Shaded}
\begin{Highlighting}[]
  \NormalTok{$ }\KeywordTok{echo} \StringTok{"*_ignored.*"} \KeywordTok{>>} \NormalTok{.gitignore}
  \NormalTok{$ }\KeywordTok{git} \NormalTok{status  }\CommentTok{# It's ignored}
  \NormalTok{$ }\KeywordTok{git} \NormalTok{commit -am }\StringTok{"add gitignore"}
\end{Highlighting}
\end{Shaded}

See \href{https://github.com/github/gitignore}{github .gitignores} You
could create global ignorance with \texttt{git config}.

Now if we want to ignore tracked files:

\begin{Shaded}
\begin{Highlighting}[]
  \NormalTok{$ }\KeywordTok{echo} \StringTok{"third.txt"} \KeywordTok{>>} \NormalTok{.gitignore}
  \NormalTok{$ }\KeywordTok{git} \NormalTok{status}
  \NormalTok{$ }\KeywordTok{echo} \StringTok{"some changes!"} \KeywordTok{>>} \NormalTok{third.txt}
  \NormalTok{$ }\KeywordTok{git} \NormalTok{status}
  \NormalTok{$ }\KeywordTok{git} \NormalTok{rm --cached third.txt}
  \NormalTok{$ }\KeywordTok{git} \NormalTok{commit -am }\StringTok{"Untrack third.txt but still available from repository"}
\end{Highlighting}
\end{Shaded}

Git ignores empty directories by default, so if you want add an empty
directory to project add \texttt{.gitkeep} to it!

\section{Go to a git trip!}\label{go-to-a-git-trip}

Every Tree-ish are a git trip ticket! But wait a min! What is a
Tree-ish:

\begin{itemize}
\itemsep1pt\parskip0pt\parsep0pt
\item
  something that references part of a tree
\item
  is \texttt{ish} because it's vary widly
\item
  is referencing a commit
\end{itemize}

Tree-ish:

\begin{itemize}
\itemsep1pt\parskip0pt\parsep0pt
\item
  Full 40 char checksum
\item
  Part of checksum (4 char at least) from left.
\item
  HEAD pointer
\item
  branch refs
\item
  tags refs
\item
  ancestry
\end{itemize}

Ancestry:

\begin{itemize}
\itemsep1pt\parskip0pt\parsep0pt
\item
  Parent commit:

  \begin{itemize}
  \itemsep1pt\parskip0pt\parsep0pt
  \item
    \texttt{HEAD\^{}}, \texttt{1f9c5bb\^{}}, \texttt{master\^{}}
  \item
    \texttt{HEAD\textasciitilde{}1}, \texttt{HEAD\textasciitilde{}}
  \end{itemize}
\item
  Grandparent Commit:

  \begin{itemize}
  \itemsep1pt\parskip0pt\parsep0pt
  \item
    \texttt{HEAD\^{}\^{}}, \texttt{1f9c5bb\^{}\^{}},
    \texttt{master\^{}\^{}}
  \item
    \texttt{HEAD\textasciitilde{}2}
  \end{itemize}
\item
  Great-grandparent commit:

  \begin{itemize}
  \itemsep1pt\parskip0pt\parsep0pt
  \item
    \texttt{HEAD\^{}\^{}\^{}}, \texttt{1f9c5bb\^{}\^{}\^{}},
    \texttt{master\^{}\^{}\^{}}
  \item
    \texttt{HEAD\textasciitilde{}3}
  \end{itemize}
\end{itemize}

\subsection{Tree-ish contents}\label{tree-ish-contents}

\begin{Shaded}
\begin{Highlighting}[]
  \NormalTok{$ }\KeywordTok{git} \NormalTok{ls-tree HEAD}
  \NormalTok{$ }\KeywordTok{git} \NormalTok{ls-tree --name-only}
  \NormalTok{$ }\KeywordTok{git} \NormalTok{show HEAD}
  \NormalTok{$ }\KeywordTok{git} \NormalTok{show CHECKSUM_OF_BLOB}
  \NormalTok{$ }\KeywordTok{git} \NormalTok{show first.txt  }\CommentTok{# ambiguous}
  \NormalTok{$ }\KeywordTok{git} \NormalTok{diff HEAD~2..HEAD}
  \NormalTok{$ }\KeywordTok{git} \NormalTok{diff --stat HEAD~2..HEAD}
  \NormalTok{$ }\KeywordTok{git} \NormalTok{diff -b TREEISH..TREEISH  }\CommentTok{# ignores space changes}
  \NormalTok{$ }\KeywordTok{git} \NormalTok{diff -w TREEISH..TREEISH  }\CommentTok{# ignores all spaces}
\end{Highlighting}
\end{Shaded}

\subsection{Logs}\label{logs}

\begin{Shaded}
\begin{Highlighting}[]
  \NormalTok{$ }\KeywordTok{git} \NormalTok{log --oneline}
  \NormalTok{$ }\KeywordTok{git} \NormalTok{log --since=}\StringTok{"2 days ago"}
  \NormalTok{$ }\KeywordTok{git} \NormalTok{log --until=2.days}
  \NormalTok{$ }\KeywordTok{git} \NormalTok{log HEAD~5..HEAD~}
  \NormalTok{$ }\KeywordTok{git} \NormalTok{log -- first.txt}
  \NormalTok{$ }\KeywordTok{git} \NormalTok{log --oneline --stat --summary}
  \NormalTok{$ }\KeywordTok{git} \NormalTok{log --format=short}
  \NormalTok{$ }\KeywordTok{git} \NormalTok{log --graph --decorate}
\end{Highlighting}
\end{Shaded}

\section{Branches}\label{branches}

\subsection{Create, Move and Delete}\label{create-move-and-delete}

\begin{Shaded}
\begin{Highlighting}[]
  \NormalTok{$ }\KeywordTok{git} \NormalTok{branch}
  \NormalTok{$ }\KeywordTok{git} \NormalTok{branch new_feature}
  \NormalTok{$ }\KeywordTok{git} \NormalTok{brance}
  \NormalTok{$ }\KeywordTok{ls} \NormalTok{-la .git/refs/heads}
  \NormalTok{$ }\KeywordTok{cat} \NormalTok{.git/refs/heads/new_feature}
  \NormalTok{$ }\KeywordTok{git} \NormalTok{checkout new_feature}
  \NormalTok{$ }\KeywordTok{cat} \NormalTok{.git/HEAD}
  \NormalTok{$ }\KeywordTok{echo} \StringTok{"in new branch"} \KeywordTok{>>} \NormalTok{second.txt}
  \NormalTok{$ }\KeywordTok{git} \NormalTok{commit -am }\StringTok{"new feature aded"}
  \NormalTok{$ }\KeywordTok{git} \NormalTok{log}
  \NormalTok{$ }\KeywordTok{git} \NormalTok{checkout master}
  \NormalTok{$ }\KeywordTok{git} \NormalTok{log}
  \NormalTok{$ }\KeywordTok{git} \NormalTok{checkout new_feature}
  \NormalTok{$ }\KeywordTok{git} \NormalTok{checkout -b feature_hotfix}
  \NormalTok{$ }\KeywordTok{echo} \StringTok{"it is too hoooooooot"} \KeywordTok{>>} \NormalTok{second.txt}
  \NormalTok{$ }\KeywordTok{git} \NormalTok{commit -am }\StringTok{"it's feel warmer now"}
  \NormalTok{$ }\KeywordTok{git} \NormalTok{checkout master}
  \NormalTok{$ }\KeywordTok{git} \NormalTok{log --oneline --graph --decorate --all}
  \NormalTok{$ }\KeywordTok{echo} \StringTok{"some annoing not commited"} \KeywordTok{>>} \NormalTok{first.txt}
  \NormalTok{$ }\KeywordTok{git} \NormalTok{checkout new_feature  }\CommentTok{# Cause error! It should be mostly clean}
  \NormalTok{$ }\KeywordTok{git} \NormalTok{reset --hard HEAD}
  \NormalTok{$ }\KeywordTok{touch} \NormalTok{temp.txt}
  \NormalTok{$ }\KeywordTok{git} \NormalTok{checkout new_feature  }\CommentTok{# this is the meaning of 'mostly clean'}
  \NormalTok{$ }\KeywordTok{git} \NormalTok{status}
  \NormalTok{$ }\KeywordTok{rm} \NormalTok{temp.txt}
  \NormalTok{$ }\KeywordTok{git} \NormalTok{diff master..new_feature}
  \NormalTok{$ }\KeywordTok{git} \NormalTok{branch --merged}
  \NormalTok{$ }\KeywordTok{git} \NormalTok{checkout master}
  \NormalTok{$ }\KeywordTok{git} \NormalTok{branch -m new_feature feature}
  \NormalTok{$ }\KeywordTok{git} \NormalTok{branch delete_me}
  \NormalTok{$ }\KeywordTok{git} \NormalTok{branch -d delete_me}
  \NormalTok{$ }\KeywordTok{git} \NormalTok{checkout -b delete_again}
  \NormalTok{$ }\KeywordTok{echo} \StringTok{"this will be deleted"} \KeywordTok{>>} \NormalTok{first.txt}
  \NormalTok{$ }\KeywordTok{git} \NormalTok{commit -am }\StringTok{"rest in peace"}
  \NormalTok{$ }\KeywordTok{git} \NormalTok{checkout master}
  \NormalTok{$ }\KeywordTok{git} \NormalTok{branch -d delete_again  }\CommentTok{# Are you sure?}
  \NormalTok{$ }\KeywordTok{git} \NormalTok{branch -D delete_again}
\end{Highlighting}
\end{Shaded}

\subsection{Merge}\label{merge}

\begin{verbatim}
      A---B---C   [feature]
     /
    D---E---F---G---H   [master]

      A---B---C   [feature]
     /         \
    D---E---F---G---H   [master]
\end{verbatim}

\begin{Shaded}
\begin{Highlighting}[]
  \NormalTok{$ }\KeywordTok{git} \NormalTok{merge feature}
  \NormalTok{$ }\KeywordTok{git} \NormalTok{merge --no-ff hotfix}
  \NormalTok{$ }\KeywordTok{git} \NormalTok{branch --merged}
\end{Highlighting}
\end{Shaded}

Strategies to avoid conflicts: - Keep lins short - keep commit short -
merge often

\section{Stash}\label{stash}

\begin{Shaded}
\begin{Highlighting}[]
  \NormalTok{$ }\KeywordTok{echo} \StringTok{"I am editing this file"} \KeywordTok{>>} \NormalTok{first.txt}
  \NormalTok{$ }\KeywordTok{git} \NormalTok{stach save }\StringTok{"some changes"}
  \NormalTok{$ }\KeywordTok{git} \NormalTok{stash list}
  \NormalTok{$ }\KeywordTok{git} \NormalTok{stash show REF}
  \NormalTok{$ }\KeywordTok{git} \NormalTok{stash show -p REF}
  \NormalTok{$ }\KeywordTok{git} \NormalTok{stash pop REF}
  \NormalTok{$ }\KeywordTok{git} \NormalTok{stash list}
  \NormalTok{$ }\KeywordTok{git} \NormalTok{stash}
  \NormalTok{$ }\KeywordTok{git} \NormalTok{stash list}
  \NormalTok{$ }\KeywordTok{git} \NormalTok{stash apply REF}
  \NormalTok{$ }\KeywordTok{git} \NormalTok{stash list}
  \NormalTok{$ }\KeywordTok{git} \NormalTok{stash drop REF}
  \NormalTok{$ }\KeywordTok{git} \NormalTok{stash clear}
\end{Highlighting}
\end{Shaded}

\section{Remotes}\label{remotes}

\begin{Shaded}
\begin{Highlighting}[]
  \NormalTok{$ }\KeywordTok{git} \NormalTok{remote add origin URL}
  \NormalTok{$ }\KeywordTok{cat} \NormalTok{.git/config}
  \NormalTok{$ }\KeywordTok{git} \NormalTok{remote}
  \NormalTok{$ }\KeywordTok{git} \NormalTok{remote -v}
  \NormalTok{$ }\KeywordTok{cat} \NormalTok{.git/config}
  \NormalTok{$ }\KeywordTok{git} \NormalTok{remote rm origin}
  \NormalTok{$ }\KeywordTok{git} \NormalTok{remote add origin URL}
  \NormalTok{$ }\KeywordTok{git} \NormalTok{push -u origin master}
  \NormalTok{$ }\KeywordTok{cat} \NormalTok{.git/config}
  \NormalTok{$ }\KeywordTok{ls} \NormalTok{.git/refs/remote}
  \NormalTok{$ }\KeywordTok{ls} \NormalTok{.git/refs/remote/origin}
  \NormalTok{$ }\KeywordTok{cat} \NormalTok{.git/refs/remote/master}
  \NormalTok{$ }\KeywordTok{git} \NormalTok{branch -r}
  \NormalTok{$ }\KeywordTok{git} \NormalTok{branch -a}
  \NormalTok{$ }\KeywordTok{git} \NormalTok{diff master origin/master}
  \NormalTok{$ }\KeywordTok{git} \NormalTok{merge origin/feature}
  \NormalTok{$ }\KeywordTok{git} \NormalTok{push origin :delete_me}
  \NormalTok{$ }\KeywordTok{git} \NormalTok{push origin --delete delete_me}
\end{Highlighting}
\end{Shaded}

\section{Rebase}\label{rebase}

\begin{verbatim}
          A---B---C   [feature]
         /
    D---E---F---G   [master]

                  A'--B'--C'   [feature]
                 /
    D---E---F---G   [master]
\end{verbatim}

\begin{Shaded}
\begin{Highlighting}[]
  \NormalTok{$ }\KeywordTok{git} \NormalTok{checkout -b extra_feature master}
  \NormalTok{$ }\KeywordTok{echo} \StringTok{"landing giant boeing"} \KeywordTok{>>} \NormalTok{first.txt}
  \NormalTok{$ }\KeywordTok{git} \NormalTok{commit -am }\StringTok{"initial commit -- extra feature"}
  \NormalTok{$ }\KeywordTok{git} \NormalTok{checkout -b hotfix_chili master}
  \NormalTok{$ }\KeywordTok{echo} \StringTok{"An chili pepper here!"} \KeywordTok{>>} \NormalTok{second.txt}
  \NormalTok{$ }\KeywordTok{git} \NormalTok{commit -am }\StringTok{"fix a hot a possible"}
  \NormalTok{$ }\KeywordTok{git} \NormalTok{checkout master}
  \NormalTok{$ }\KeywordTok{git} \NormalTok{merge hotfix_chili}
  \NormalTok{$ }\KeywordTok{git} \NormalTok{branch -d hotfix_chili}
  \NormalTok{$ }\KeywordTok{git} \NormalTok{log --oneline --decorate}
  \NormalTok{$ }\KeywordTok{git} \NormalTok{checkout extra_feature}
  \NormalTok{$ }\KeywordTok{git} \NormalTok{log --oneline --decorate}
  \NormalTok{$ }\KeywordTok{git} \NormalTok{rebase master}
  \NormalTok{$ }\KeywordTok{git} \NormalTok{log --oneline --decorate}
  \NormalTok{$ }\KeywordTok{git} \NormalTok{reflog}
  \NormalTok{$ }\KeywordTok{git} \NormalTok{reset --hard HEAD@}\DataTypeTok{\{NUM\}}
  \NormalTok{$ }\KeywordTok{git} \NormalTok{rebase -i master}
  \NormalTok{$ }\KeywordTok{git} \NormalTok{push -f origin extra_feature  }\CommentTok{# should push forcefully}
\end{Highlighting}
\end{Shaded}

The Golden Rule of Rebasing: \textbf{Once you understand what rebasing
is, the most important thing to learn is when not to do it. The golden
rule of git rebase is to never use it on public branches.} see:
\href{https://www.atlassian.com/git/tutorials/merging-vs-rebasing}{Merge
vs Rebase}

\section{Tagging}\label{tagging}

\begin{Shaded}
\begin{Highlighting}[]
  \NormalTok{$ }\KeywordTok{git} \NormalTok{log --oneline --decorate}
  \NormalTok{$ }\KeywordTok{git} \NormalTok{tag v0.1.0}
  \NormalTok{$ }\KeywordTok{git} \NormalTok{tag}
  \NormalTok{$ }\KeywordTok{git} \NormalTok{show v0.1.0}
  \NormalTok{$ }\KeywordTok{git} \NormalTok{tag -a v0.1.0rc HAED~1 -m }\StringTok{"First RC realse"}
  \NormalTok{$ }\KeywordTok{git} \NormalTok{log --oneline --decorate}
  \NormalTok{$ }\KeywordTok{git} \NormalTok{reflog}
  \NormalTok{$ }\KeywordTok{git} \NormalTok{push origin --tags}
\end{Highlighting}
\end{Shaded}

\section{Git workflow}\label{git-workflow}

see:
\href{https://www.atlassian.com/git/tutorials/comparing-workflows/}{Comparing
Workflows}

\subsection{Centralized Workflow}\label{centralized-workflow}

\begin{itemize}
\itemsep1pt\parskip0pt\parsep0pt
\item
  Everyone works on master!
\item
  Needs many rebases
\item
  Confilicts are totaly accepted!
\item
  Only one branch make the way clean (!?)
\end{itemize}

\subsection{Feature Branch Workflow}\label{feature-branch-workflow}

\begin{itemize}
\itemsep1pt\parskip0pt\parsep0pt
\item
  Every feature have thier own brach
\item
  Based on pull requests
\item
  Less confilicts may happens
\item
  There may be many waits for a branch owner to finish their work
\end{itemize}

\subsection{Gitflow Workflow}\label{gitflow-workflow}

see: \href{http://nvie.com/posts/a-successful-git-branching-model/}{A
successful Git branching model}

\begin{itemize}
\itemsep1pt\parskip0pt\parsep0pt
\item
  There are 5 basic branch type:
\item
  master: stable release
\item
  develop: main development branch
\item
  feature: branch for new feature development
\item
  release: branch from develop for every single feature releases
  e.g.~1.5.2 -\textgreater{} 1.6 not 1.5.3 nor 2.0
\item
  hotfix: branch from master to apply hotfixes to master
\item
  Compatible with \href{http://semver.org/}{semver}
\item
  Rare confilicts
\item
  Descentralized while Centralized
\item
  Very collaborative atmosphere
\item
  Never breaks stable releases
\item
  A bit complex to handle, nevertheless there is a handy automation tool
\end{itemize}

\href{https://github.com/nvie/gitflow}{git-flow} is a tool to automate
this model. see:
\href{http://danielkummer.github.io/git-flow-cheatsheet/}{git-flow
cheatsheet}

\subsection{Forking Workflow}\label{forking-workflow}

\begin{itemize}
\itemsep1pt\parskip0pt\parsep0pt
\item
  Every member works on thier own copy of repository (fork)
\item
  Based on pull requests
\item
  There is a maintainer responsible to merge everything
\item
  Prevents unwanted changes to main repository
\item
  Usually opensource/public projects use this model
\end{itemize}

\subsection{Patch Workflow}\label{patch-workflow}

see: \href{http://rypress.com/tutorials/git/patch-workflows}{Patch
Workflows}

\begin{itemize}
\itemsep1pt\parskip0pt\parsep0pt
\item
  Every member works on thier own copy of repository (fork)
\item
  Based on emailed patches! (see:
  \href{https://git-scm.com/book/en/v2/Git-Commands-Email}{A3.9 Git
  Commands - Email})
\item
  There is a maintainer responsible to merge everything
\item
  Linux Kernel uses this model!
\end{itemize}

\end{document}
